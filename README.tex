\documentclass{article}
\usepackage[xetex,
  pdfborder={0 0 0},
  colorlinks,
  linkcolor=blue,
  citecolor=blue,
  urlcolor=blue,
]{hyperref}
\title{Four Frequency-Strategy Solutions to Hangman}
\author{Peter Danenberg
  \texttt{<}\href{mailto:danenberg@post.harvard.edu}
  {\nolinkurl{danenberg@post.harvard.edu}}\texttt{>}}
\begin{document}
\maketitle
\begin{abstract}
I'd like to present four solutions to the hangman problem which are
sub-strategies of an arch-strategy called \emph{frequency-strategy}:
it reduces the word-space by filtering on the most common unguessed
letter. Three sub-strategies, the \emph{regex-strategy},
\emph{predicate-strategy} and \emph{trie-strategy}, which differ by
how they calculate the most common letter, are discussed below. There
are also \emph{deterministic} and \emph{sampling} variants on
these sub-strategies.

The strategies in order of decreasing performance are: deterministic
regex-strategy; sampling regex-strategy; predicate-strategy;
trie-strategy.
\end{abstract}
\tableofcontents
\section{Frequency Strategy}
The frequency strategy reduces the space of possible solutions by
filtering on the most common letter, guessing a word when the
remaining-words/remaining-guesses ratio looks auspicious:

\begin{enumerate}
\item \label{filter} Filter on the last guessed letter (if it exists).
\item Is the ratio of words to remaining guesses auspicious?
  \begin{enumerate}
  \item If so, guess a remaining word.
  \item \label{common-letter} Otherwise, guess the most common unguessed letter.
  \end{enumerate}
\item Is the game over?
  \begin{enumerate}
  \item If so, return the score.
  \item Otherwise, return to step \ref{filter}.
  \end{enumerate}
\end{enumerate}

\section{Substrategies}
The substrategies of the frequency strategy are variations

\subsection{Trie strategy}
\subsection{Predicate strategy}
\subsection{Regex strategy}
\subsubsection{Deterministic regex strategy}
\subsubsection{Sampling regex strategy}
\section{Invoking \texttt{hangman}}
\section{Unimplemented Possible Improvements}
\begin{itemize}
\item Don't merely guess a remaining word randomly: rank them by
  frequency of constituent letters.
\end{itemize}
\end{document}
