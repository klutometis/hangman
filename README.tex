\documentclass{article}
\usepackage[autosize]{dot2texi}
\usepackage{tikz}
\usetikzlibrary{automata,snakes,arrows,shapes}
\usepackage{float}
\usepackage[xetex,
  pdfborder={0 0 0},
  colorlinks,
  linkcolor=blue,
  citecolor=blue,
  urlcolor=blue,
]{hyperref}
\title{Four Frequency-Strategy Solutions to Hangman}
\author{Peter Danenberg
  \texttt{<}\href{mailto:danenberg@post.harvard.edu}
  {\nolinkurl{danenberg@post.harvard.edu}}\texttt{>}}
\begin{document}
\maketitle
\begin{abstract}
I'd like to present four solutions to the hangman problem which are
sub-strategies of an arch-strategy called \emph{frequency-strategy}:
it reduces the word-space by filtering on the most common unguessed
letter. Three sub-strategies, the \emph{regex-strategy},
\emph{predicate-strategy} and \emph{trie-strategy}, which differ by
how they calculate the most common letter, are discussed below. There
are also \emph{deterministic} and \emph{sampling} variants on
these sub-strategies.

The strategies in order of decreasing performance are: deterministic
regex-strategy; sampling regex-strategy; predicate-strategy;
trie-strategy.
\end{abstract}
\tableofcontents
\section{Frequency Strategy}
The frequency strategy reduces the space of possible solutions by
filtering on the most common letter, guessing a word when the
remaining-words/remaining-guesses ratio looks auspicious:

\begin{enumerate}
\item \label{filter} Filter on the last guessed letter (if it exists).
\item Is the ratio of words to remaining guesses auspicious?
  \begin{enumerate}
  \item If so, guess a remaining word.
  \item \label{common-letter} Otherwise, guess the most common unguessed letter.
  \end{enumerate}
\item Is the game over?
  \begin{enumerate}
  \item If so, return the score.
  \item Otherwise, return to step \ref{filter}.
  \end{enumerate}
\end{enumerate}

\section{Substrategies}
The sub-strategies of the frequency arch-strategy are variations on
step \ref{common-letter}.

\subsection{Trie strategy}
The trie-strategy encodes the dictionary in a sparse, 26-ary trie (see
figure \ref{trie-graph}). I had high hopes for the trie-strategy because
of the $O(\log_{26} n)$ lookup and deletion; it both performed the
worst, however, and was least accurate.

I suspect that, on the one hand, constant factors intervened; on the
other, I decided to sacrifice certain trie-invariants for the sake of
performance. As a result of pathological trie-degeneration, the
letter-counting algorithm counted spurious letters; and adjusting the
algorithm to maintain trie-invariants gave even worse performance.

\begin{figure}[H]
  \begin{center}
    \begin{dot2tex}[fdp]
      digraph G {
        a1 [label=a]
        a2 [label=a]
        r1 [label=r]
        r2 [label=r]
        // r3 [label=r]
        g1 [label=g]
        g2 [label=g]
        h1 [label=h]
        h2 [label=h]
        h3 [label=h]
        // h4 [label=h]
        s1 [label=s]
        s2 [label=s]
        s3 [label=s]
        i1 [label=i]
        e1 [label=e]
        // e2 [label=e]
        root -> a1 -> a2 -> r1 -> g1 -> h1
        r1 -> r2 -> g2 -> h2 -> h3
        root -> z -> y -> m -> o -> s1 -> i1 -> s2
        s1 -> e1 -> s3
        // root -> c -> h4 -> e2 -> r3 -> u -> b
      }
    \end{dot2tex}
  \end{center}
  \caption{The sparse 26-ary trie encoding `aargh', `aarrgh',
    `aarrghh', `zymoses' and `zymosis'.}
  \label{trie-graph}
\end{figure}

\subsection{Predicate strategy}

The predicate-strategy is a linear strategy which is less general than
the regex-strategy below: it filters integer-encoded
string-representations with ad-hoc negative and positive predicates.

\subsection{Regex strategy}
\subsubsection{Deterministic regex strategy}
\subsubsection{Sampling regex strategy}
\section{Invoking \texttt{hangman}}
\section{Unimplemented Possible Improvements}
\begin{itemize}
\item Don't merely guess a remaining word randomly: rank them by
  frequency of constituent letters.
\end{itemize}
\end{document}
