\documentclass{article}
\usepackage[autosize]{dot2texi}
\usepackage{tikz}
\usepackage{amsmath}
\usetikzlibrary{automata,snakes,arrows,shapes}
\usepackage{float}
\usepackage{booktabs}
\usepackage[xetex,
  pdfborder={0 0 0},
  colorlinks,
  linkcolor=blue,
  citecolor=blue,
  urlcolor=blue,
]{hyperref}
\title{Four Frequency-Strategy Solutions to Hangman}
\author{Peter Danenberg
  \texttt{<}\href{mailto:danenberg@post.harvard.edu}
  {\nolinkurl{danenberg@post.harvard.edu}}\texttt{>}}
\begin{document}
\maketitle
\begin{abstract}
I'd like to present four solutions to the hangman problem which are
sub-strategies of an arch-strategy called \emph{frequency-strategy}:
frequency-strategy reduces the word-space by filtering on the most
common unguessed letter and guesses words when the
remaining-words/remaining-guesses ratio looks auspicious. Three
sub-strategies, the \emph{regex-strategy}, \emph{predicate-strategy}
and \emph{trie-strategy}, which differ by how they calculate the most
common letter, are discussed below. There are also
\emph{deterministic} and \emph{sampling} variants on these
sub-strategies.

The strategies in order of decreasing performance are: deterministic
regex-strategy; sampling regex-strategy; predicate-strategy;
trie-strategy.
\end{abstract}
\tableofcontents
\section{Frequency Strategy}
The frequency strategy reduces the space of possible solutions by
filtering on the most common letter, guessing a word when the
remaining-words/remaining-guesses ratio looks auspicious:

\begin{enumerate}
\item \label{filter} Filter on the last guessed letter (if it exists).
\item Is the ratio of words to remaining guesses auspicious?
  \begin{enumerate}
  \item If so, guess a remaining word.
  \item \label{common-letter} Otherwise, guess the most common unguessed letter.
  \end{enumerate}
\item Is the game over?
  \begin{enumerate}
  \item If so, return the score.
  \item Otherwise, return to step \ref{filter}.
  \end{enumerate}
\end{enumerate}

\section{Substrategies}
The sub-strategies of the frequency arch-strategy are variations on
step \ref{common-letter}; table \ref{performance} has a comparison of
their relative performance when run over the entire dictionary.

\begin{table}
  \begin{center}
    \begin{tabular}{lcccc}
      \toprule
      & Trie & Predicate & Determ. rgx. & Sampl. rgx. \\
      \midrule
      Average time (ms) & 315.19897 & 69.23657 & 22.884363 & 22.892752 \\
      Average score & 11.089 & 7.214 & 7.082 & 7.077 \\
      \bottomrule
    \end{tabular}
  \end{center}
  \caption{Relative performance of the trie-, predicate-,
    deterministic- and sampling-regex-strategies over $1000$ random
    words.}
  \label{performance}
\end{table}

\subsection{Trie strategy}

The trie-strategy encodes the dictionary in a sparse, 26-ary trie (see
figure \ref{trie-graph}). I had high hopes for the trie-strategy because
of the $O(\log_{26} n)$ lookup and deletion; it both performed the
worst, however, and was least accurate.

\begin{figure}[h]
  \begin{center}
    \begin{dot2tex}[fdp]
      digraph G {
        a1 [label=a]
        a2 [label=a]
        r1 [label=r]
        r2 [label=r]
        g1 [label=g]
        g2 [label=g]
        h1 [label=h]
        h2 [label=h]
        h3 [label=h]
        s1 [label=s]
        s2 [label=s]
        s3 [label=s]
        i1 [label=i]
        e [label=e]
        root -> a1 -> a2 -> r1 -> g1 -> h1
        r1 -> r2 -> g2 -> h2 -> h3
        root -> z -> y -> m -> o -> s1 -> i1 -> s2
        s1 -> e -> s3
      }
    \end{dot2tex}
  \end{center}
  \caption{The sparse 26-ary trie encoding `aargh', `aarrgh',
    `aarrghh', `zymoses' and `zymosis'.}
  \label{trie-graph}
\end{figure}

I suspect that, on the one hand, constant factors intervened; on the
other, I decided to sacrifice certain trie-invariants for the sake of
performance. As a result of pathological trie-degeneration, the
letter-counting algorithm counted spurious letters; and adjusting the
algorithm to maintain trie-invariants gave even worse performance.

\begin{description}
  \item [Average time (ms)] 315.19897
  \item [Average score] 11.089
\end{description}

\subsection{Predicate strategy}

The predicate-strategy is a linear strategy which is less general than
the regex-strategy below: it filters integer-encoded\footnote{The
  integers are simply $\text{ASCII}(letter) - 97$.}
string-representations (see figure \ref{predicate-strings}) with
ad-hoc negative and positive predicates. The predicates corresponding
to a game-state of e.g. `\texttt{\_\_R\_\_}' with a last guess of
`\texttt{E}' (4) are $(\lnot 4 \mathbin{\land} \lnot 4 \mathbin{\land}
17 \mathbin{\land} \lnot 4 \mathbin{\land} \lnot 4)$.

\begin{figure}[h]
  \begin{center}
    \begin{texttt}
      (0 0 17 6 7) \\
      (0 0 17 17 6 7) \\
      (0 0 17 17 6 7 7) \\
      (25 24 12 14 18 4 18) \\
      (25 24 12 14 18 8 18)
    \end{texttt}
  \end{center}
  \caption{Predicate-encoding for `aargh', `aarrgh', `aarrghh',
    `zymoses' and `zymosis'.}
  \label{predicate-strings}
\end{figure}

Despite the fact that these ad-hoc predicates are less general than
regular expressions, they fail to outperform them; I suspect that the
overhead of \emph{lambda}-application is to blame.

\begin{description}
  \item [Average time (ms)] 69.23657
  \item [Average score] 7.214
\end{description}

\subsection{Regex strategy}

The regex-strategy, like the predicate strategy, is linear; it stores
it dictionaries, however, as lists of character-encoded strings %% (see
%% figure \ref{regex-strings})
and filters them with negative and positive regular expressions. The
regex corresponding to a game-state of e.g. `\texttt{\_\_R\_\_}' with
a last guess of `\texttt{E}' is
\texttt{\^{}[\^{}E][\^{}E]R[\^{}E][\^{}E]\$}.

\subsubsection{Sampling regex strategy}

The sampling regex strategy differs from its deterministic sibling in
that, instead of making an exhaustive count of letters frequencies, it
samples the word space until a stable ratio of letters appears. The
notion of stability has three tuning parameters:

\begin{enumerate}
\item $\epsilon$, or \texttt{delta-percentage-tolerance}, the
  percentage below which changes in frequency are considered
  negligible;
\item \texttt{sampling frequency}, the number of iteration between
  sampling; and
\item \texttt{ratio-to-n-of-minimum-iterations}, the ratio of the
  total words; after which to begin sampling.
\end{enumerate}

It should be possible to tune these parameters to outperform
deterministic regex with some sacrifice in accuracy.

\begin{description}
  \item [Average time (ms)] 22.892752
  \item [Average score] 7.077
\end{description}

%% \begin{figure}[h]
%%   \begin{center}
%%     \begin{texttt}
%%       aargh \\
%%       aarrgh \\
%%       aarrghh \\
%%       zymoses \\
%%       zymosis
%%     \end{texttt}
%%   \end{center}
%%   \caption{Regex-encoding for `aargh', `aarrgh', `aarrghh',
%%     `zymoses' and `zymosis'.}
%%   \label{regex-strings}
%% \end{figure}

\subsubsection{Deterministic regex strategy}

The deterministic regex-strategy differs from its sampling sibling in
that it does a full frequency count of all letters before making a
guess. It should, in theory, be more accurate than sampling and
slightly slower; in practice, it is practically indistinguishable from
sampling with respect to speed and score. In table \ref{performance},
for instance, is happened to be both faster than sampling and less
accurate.

\begin{description}
  \item [Average time (ms)] 22.884363
  \item [Average score] 7.082
\end{description}

\section{Invoking \texttt{hangman}}
\section{Conclusions and Possible Improvements}

I'm a little ashamed to admit that I couldn't beat linear regex, even
with a more sophisticated trie-algorithm; it may be that I shouldn't
have been so dogmatic about using functional algorithms: a little
mutation doesn't hurt once in a while.

Some TODOs:

\begin{itemize}
\item Don't merely guess a remaining word randomly: rank them by
  frequency of constituent letters.
\item Tune sampling regex so that it outperforms deterministic regex.
\item Beat linear regex!
\end{itemize}
\end{document}
